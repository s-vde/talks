\documentclass[]{article}

\newtheorem{definition}{Definition}

\begin{document}
\title{Verifying Multi-Threaded Programs Using Dynamic Analysis}
\author{Susanne van den Elsen}
\date{November 10, 2017}
\maketitle

% 5850 words for 45 minutes (130 per minute)

\section*{Instroduction}
\paragraph{Multi-Threading at Native Instruments}
I am a software developer at Native Instruments in Berlin. 
At Native Instruments we create software and hardware for computer-based audio production.
Before that, I was doing research at the Max Planck Institute for Software Systems.
My research interests were in automated software verification, and one of my projects focused specifically on
verification of multi-threaded software.
Multi-threading is very relevant in audio production software.
First of all, the audio processing should always run in order not to produce glitchy artifacts. 
Therefore, it should run on a dedicated, high priority thread.
It should never wait for something else in the program.
At the same time, it needs to communicate with the user interface, for example to receive parameter changes.
Second, audio processing itself, potentially running complex DSP algorithms, is generally CPU-heavy and often needs to 
exploit the power of multiple cores.
In short, we need to address challenges in performance and correctness related to multi-threading on a daily basis.

\paragraph{}
There are many good conference talks about what is multi-threading, when and when not to use it and when you use it, 
how to do it right.
In this talk I will focus on techniques for automatically verifying that a given multi-threaded program is correct.
I will present a tool I developed in the context of my research and give a very high-level explanation of the theory
behind it.


\section*{Traditional Testing / Verification Tools}
Before diving into solutions, let me first explain why testing and verification tools designed for single-threaded 
programs are not sufficient for testing or verifying multi-threading programs.

\paragraph{Concurrency Errors}
First of all, multi-threading introduces a whole new set of potential bugs.
These generally arise from the fact that threads can operate on shared resources.
\begin{definition}[Race Condition]
A \emph{race condition} occurs when the \emph{timing} or \emph{ordering} of events may lead to erroneous program 
behavior. It is therefore a purely \emph{semantic} error.
\end{definition}
\begin{definition}[Data Race]
A \emph{data race} occurs when two instructions performed by different threads access the same memory location, at 
least one of the accesses is modifying the data at that location and the two instructions are not related by the 
\emph{happens-before relation}. Intuitively, this means that they may occur \emph{concurrently} and it is undefined
which of the accesses will become visible first.
\end{definition}
\begin{definition}[Dead Lock]
A \emph{deadlock} occurs when two or more threads cannot make progress because they are waiting for each other to 
release a shared resource.
\end{definition}


\paragraph{Concurrency}
Concurrency is notoriously hard to get right.
The difficulty lies in reasoning about the possible interactions between threads or processes.
These interactions may introduce subtle bugs like race conditions, data races and deadlocks.

\paragraph{Debugging}
Debugging multi-threaded programs is challenging as well.
Trace analysis tools, like ThreadSanitizer and Helgrind, deploy algorithms specifically tailored to detect data races 
and deadlocks from a program's execution trace.
Using these tools, many bugs in real-world software have been detected.
However, trace analysis by itself cannot provide any guarantees about the bug-freedom of a program.
The behavior of the program is largely determined by the order in which threads are scheduled or in which their
actions become visible to other threads.
A potential bug may only manifest itself in a single, very rare scheduling, which happens not to occur during 
testing, leaving the bug undetected and unsolved.
Even when detected, since the scheduling is not under the programmer's control, a bug is not easily reproducible.

\paragraph{Systematic Exploration}
Systematic exploration is a technique to overcome these challenges.
It involves taking control over the scheduler and repeatedly executing a program under test under a different, 
unexplored schedule.
Schedules leading to buggy executions can be recorded and replayed at any time.

The most straightforward way to do systematic exploration is to do a depth-first exploration of the tree corresponding
to a program's set of possible schedules.
This variant is complete, that is, all behaviors of the program are seen and all bugs detected.
However, it suffers from combinatorial explosion; the number of possible schedules grows exponentially with the size
of the program.
Therefore, for nontrivial programs, this method is not feasible.

The problem of combinatorial explosion can be tackled by carefully pruning the tree of schedules that are considered for
exploration.
This introduces a trade-off between the number of explored schedules and the strength of the provided correctness 
guarantees.

\paragraph{Preemption Bounded Search}
Preemption bounded search reduces the schedules considered by limiting the number of preemptive context switches that
are allowed in a schedule.
Small bounds greatly reduce the state-space.
On the other hand, the method is not complete.
It provides bounded coverage guarantees: all remaining, undiscovered bugs require $c+1$ preemptions to manifest
themselves.
The method can be run iteratively with increasing bound, so that \emph{in the limit} all schedules are explored.
In short, this method allows to trade-off resource budget for coverage.
Also, it tends to produce counterexamples that are easy to understand.
Empirical study has shown that many bugs can be indeed found with low preemption bound [1,2].

\paragraph{Partial Order Reduction}
The high level idea behind partial order reduction (POR) is to group together schedules that induce equivalent behavior 
in equivalence classes. Then, ideally only a single schedule per equivalence class is explored. An often used 
equivalence relation, the happens-before relation, is the reflexive, transitive closure of the dependence
relation on transitions.

\section*{Bibliography}
\begin{description}
\item[1]
M. Musuvathi and S. Qadeer.  
Iterative context bounding for systematic testing of multithreaded programs. 
In PLDI, pages 446–455, 2007.
%https://www.microsoft.com/en-us/research/wp-content/uploads/2016/02/pldi07-icb.pdf

\item[2]
%https://people.engr.ncsu.edu/gjin2/Classes/591/Spring2017/concurrency-schedule-bounding.pdf
%https://github.com/mc-imperial/sctbench
\end{description}


\section*{Notes}
Dynamic analysis is a technique in which a program under test is run and a trace of its execution is analyzed to detect 
potential errors.

Automated verification is the task of, given an input program and a specification, algorithmically decide whether the 
program satisfies the specification.
The verification task is undecidable in general.
Therefore, efforts restrict either the class of input programs, the language in which specifications can be expressed, 
or both.
In this talk, I focus on multi-threaded programs that are guaranteed to terminate and specification that the program
is datarace and deadlock free.



for example by at each 
point forcing the operating system's scheduler to schedule a given thread.
It performs trace analysis on the execution and then computes the next schedule to run the program through.

Delay bounding: limiting the number of times a schedule can deviate from a given deterministic scheduler.




\section{Definitions}


\end{document}
